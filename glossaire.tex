\newglossaryentry{DataLab}
{
    name=DataLab,
    description={Un datalab est un espace exclusivement dédié à l’expérimentation et à la qualification « fonctionnelle » des différentes données de l’entreprise. En effet, il permet d’explorer des jeux de données, de les traiter, mais aussi de mettre à l’épreuve des algorithmes de Machine Learning.}
}
\newglossaryentry{forge logicielle}
{
    name= Forge logicielle,
    description={Une forge logicielle est une plateforme collaborative accessible par le Web permettant de travailler conjointement sur un même projet}
}
\newglossaryentry{batchs}
{
    name=Batch,
    description={Un batch désigne l’automatisation d’une suite de commandes exécutées en série sur un ordinateur sans qu’il soit nécessaire qu’un opérateur intervienne pour réaliser cette opération}
}
\newglossaryentry{Puppet}
{
    name=Puppet,
    description={Logiciel libre permettant la gestion de la configuration de serveurs}
}
\newglossaryentry{Rundeck}
{
    name=Rundeck,
    description={Logiciel libre permettant l’automatisation d’administration de serveurs via la création de jobs ou tâches}
}
\newglossaryentry{ISO}
{
    name=ISO,
    description={Une image disque (ou image ISO) est un (voire plusieurs) fichier(s) archive proposant la copie conforme d'un disque optique ou magnétique (tel qu'il serait écrit sur celui-ci).}
}
\newglossaryentry{Image}
{
    name=Image,
    description={Une image de conteneur est un fichier statique non modifiable. Il renferme un code exécutable de manière à exécuter un processus isolé dans une infrastructure informatique.}
}
\newglossaryentry{Réplica}
{
    name=Réplica,
    description={Nombre d’instance d’une application}
}
\newglossaryentry{Moteur}
{
    name=Moteur d'éxécution,
    description={Un environnement d’exécution ou runtime est un logiciel responsable de l’exécution des programmes informatiques écrits dans un langage de programmation donné}
}
\newglossaryentry{Healthcheck}
{
    name=Healthcheck,
    description={Contrôle servant à vérifier si l’application s’exécute correctement}
}
\newglossaryentry{Service Discovery}
{
    name= Service Discovery,
    description={Protocole réseau permettant la découverte de service}
}
\newglossaryentry{Cgroups}
{
    name=Cgroups,
    description={cgroups (control groups) est une fonctionnalité du noyau Linux pour limiter, compter et isoler l’utilisation desressources (processeur, mémoire, utilisation disque, etc.)}
}
\newglossaryentry{Scalabilité}
{
    name=Scalabilité: ,
    description={(littéralement changer d'échelle) En informatique matérielle et logicielle et en télécommunications, l’extensibilité ou scalabilité désigne la capacité d'un produit à s'adapter à un changement d'ordre de grandeur de la demande (montée en charge), en particulier sa capacité à maintenir ses fonctionnalités et ses performances en cas de forte demande. On distingue la scalabilité verticale (augmentation des ressources allouées :  CPU, mémoire...) et scalabilité horizontale (augmentation du nombre de répliques du produits).}
}
\newglossaryentry{Commits}
{
    name=Commit,
    description={Un commit est le fait d’enregistrer dans un outil de gestion de versions (par exemple GitLab) une nouvelle version d’un ensemble de fichiers}
}
\newglossaryentry{Pod}
{
    name=Pod,
    description={Un Pod est la plus petite unité logique de k8s, encapsule un ou plusieurs conteneurs (dans ce cas il partage lemême contexte) Un Pod est attaché à un node, son existence est temporaire, on peut facilement le répliquer sur un autrenode. Chaque Pod a une adresse IP virtuelle unique au sein du cluster ce qui permettra aux autres Pods du cluster decommuniquer avec ce dernier.}
}
\newglossaryentry{Node}
{
    name=Node,
    description={(Noeud) terme utilisé pour désigné une machine appartenant à un cluster. On en distingue en général 3 types: les worker nodes, les public nodes et les master nodes.}
}
\newglossaryentry{Service}
{
    name=Service: ,
    description={Un service peut être défini comme un ensemble logique de Pods exposés en tant que service réseau. C’est unniveau d’abstraction au-dessus du Pod, qui fournit une adresse IP et un nom DNS unique pour un ensemble de Pods.}
}
\newglossaryentry{Deployment}
{
    name=Deployment: ,
    description={}
}
\newglossaryentry{Ingress}
{
    name=Ingress: ,
    description={}
}
\newglossaryentry{Charts}
{
    name=Charts: ,
    description={ollection de fichiers YAML variabilisés décrivant des ressources qui, misent bout à bout, donnent une application déployable sur Kubernetes}
}
\newglossaryentry{Releases}
{
    name=Releases: ,
    description={Version d'un chart déployée dans kubernetes}
}
\newglossaryentry{CNCF}
{
    name=CNCF,
    description={Cloud native computing foundation (CNCF) est un organisme recommandant les bonnes pratiques dans la mise en place de plateformes Cloud.}
}
\newglossaryentry{Template}
{
    name=Template: ,
    description={Un template est un modèle générique dans lequel on va pouvoir insérer des modifications. Dans le cadre de ce rapport il faut le voir comme une description d'une application avec des trous que l'on remplira par la suite.}
}
\newglossaryentry{Go}
{
    name=Go: ,
    description={Go est un langage de programmation compilé et concurrent inspiré de C et Pascal.}
}
\newglossaryentry{Merge Request}
{
    name=Merge Request,
    description={(aussi appelé demande de fusion) Déclaration d'une demande de fusion entre le code nouvellement développé et le code déjà présent sur le répertoire (Git).}
}
\newglossaryentry{gRPC}
{
    name=,
    description={}
}
\newglossaryentry{RollBack}
{
    name=,
    description={}
}
\newglossaryentry{CloudNative}
{
    name=,
    description={}
}
\newglossaryentry{Stateless}
{
    name=,
    description={}
}
\newglossaryentry{}
{
    name=,
    description={}
}
\newglossaryentry{}
{
    name=,
    description={}
}
\newglossaryentry{}
{
    name=,
    description={}
}
\newglossaryentry{}
{
    name=,
    description={}
}
\newglossaryentry{}
{
    name=,
    description={}
}


