\newglossaryentry{DataLab}
{
    name=DataLab : ,
    description={Un datalab est un espace exclusivement dédié à l’expérimentation et à la qualification « fonctionnelle » des différentes données de l’entreprise. En effet, il permet d’explorer des jeux de données, de les traiter, mais aussi de mettre à l’épreuve des algorithmes de Machine Learning}
}
\newglossaryentry{forge logicielle}
{
    name=Forge logicielle : ,
    description={Une forge logicielle est une plateforme collaborative accessible par le Web permettant de travailler conjointement sur un même projet}
}
\newglossaryentry{batchs}
{
    name=Batch : ,
    description={Un batch désigne l’automatisation d’une suite de commandes exécutées en série sur un ordinateur, sans qu’il soit nécessaire qu’un opérateur intervienne pour réaliser cette opération}
}
\newglossaryentry{Puppet}
{
    name=Puppet : ,
    description={Logiciel libre permettant la gestion de la configuration de serveurs}
}
\newglossaryentry{Rundeck}
{
    name=Rundeck : ,
    description={Logiciel libre permettant l’automatisation d’administration de serveurs via la création de jobs ou tâches}
}

\newglossaryentry{Image}
{
    name=Image : ,
    description={Une image d'un conteneur est un fichier statique non modifiable. Il renferme un code exécutable de manière à exécuter un processus isolé dans une infrastructure informatique}
}
\newglossaryentry{Réplica}
{
    name=Réplica : ,
    description={Nombre d’instance d’une application}
}
\newglossaryentry{Moteur}
{
    name=Moteur d'éxécution : ,
    description={Un environnement d’exécution ou runtime est un logiciel responsable de l’exécution des programmes informatiques écrits dans un langage de programmation donné}
}
\newglossaryentry{Healthcheck}
{
    name=Healthcheck : ,
    description={Contrôle servant à vérifier si l’application s’exécute correctement}
}
\newglossaryentry{Service Discovery}
{
    name=Service Discovery : ,
    description={Protocole réseau permettant la découverte de service}
}
\newglossaryentry{Cgroups}
{
    name=Cgroups : ,
    description={Cgroups (control groups) est une fonctionnalité du noyau Linux pour limiter, compter et isoler l’utilisation desressources (processeur, mémoire, utilisation disque, etc.)}
}
\newglossaryentry{Scalabilité}
{
    name=Scalabilité : ,
    description={(littéralement changer d'échelle) En informatique matérielle et logicielle et en télécommunications, l’extensibilité ou scalabilité désigne la capacité d'un produit à s'adapter à un changement d'ordre de grandeur de la demande (montée en charge), en particulier sa capacité à maintenir ses fonctionnalités et ses performances en cas de forte demande. On distingue la scalabilité verticale (augmentation des ressources allouées :  CPU, mémoire...) et scalabilité horizontale (augmentation du nombre de répliques du produits)}
}
\newglossaryentry{Commits}
{
    name=Commit : ,
    description={Un commit est le fait d’enregistrer dans un outil de gestion de versions (par exemple GitLab) une nouvelle version d’un ensemble de fichiers}
}

\newglossaryentry{Push}
{
    name=Push : ,
    description={Push, dans le cadre d'un outil de gestion de versions comme Git, signifie envoyer les modifications du répertoire local au répertoire distant (Git).}
}

\newglossaryentry{Pod}
{
    name=Pod : ,
    description={Un Pod est la plus petite unité logique de k8s. Il encapsule un conteneur applicatif. Un Pod est attaché à un node, son existence est temporaire, on peut facilement le répliquer sur un autre node. Chaque Pod a une adresse IP virtuelle unique au sein du cluster, ce qui permettra aux autres Pods du cluster de communiquer avec ce dernier}
}
\newglossaryentry{Node}
{
    name=Node : ,
    description={Node signifie nœud en français. Un node désigne une machine appartenant à un cluster. On en distingue en général trois types : les worker nodes (nœuds de travail), les public nodes (nœuds publics) et les master nodes (nœuds maîtres)}
}
\newglossaryentry{Service}
{
    name=Service : ,
    description={Un service peut être défini comme un ensemble logique de Pods exposés en tant que service réseau. C’est un niveau d’abstraction au-dessus du Pod, qui fournit une adresse IP et un nom DNS unique pour un ensemble de Pods}
}
\newglossaryentry{Deployment}
{
    name=Deployment : ,
    description={L'objet deployment (déploiement en français) est un objet kubernetes fournissant des mises à jour déclaratives pour Pods et ReplicaSet}
}
\newglossaryentry{Ingress}
{
    name=Ingress : ,
    description={L'ingress (en français l'entrée) est un objet kubernetes qui gère l'accès externe aux services du cluster, généralement par HTTP. L'ingress gère l'équilibrage de charge (load balancing), La gestion du https}
}
\newglossaryentry{Charts}
{
    name=Charts : ,
    description={Collection de fichiers YAML variabilisés décrivant des ressources qui, misent bout à bout, donnent une application déployable sur Kubernetes}
}
\newglossaryentry{Releases}
{
    name=Releases : ,
    description={Version d'un chart déployée dans kubernetes}
}
\newglossaryentry{CNCF}
{
    name=CNCF : ,
    description={Cloud native computing foundation (CNCF) est un organisme recommandant les bonnes pratiques dans la mise en place de plateformes Cloud}
}
\newglossaryentry{Template}
{
    name=Template : ,
    description={Un template est un modèle générique dans lequel on va pouvoir insérer des modifications. Dans le cadre de ce rapport il faut le voir comme une description d'une application avec des trous que l'on remplira par la suite}
}
\newglossaryentry{Go}
{
    name=Go : ,
    description={Go est un langage de programmation compilé et concurrent inspiré de C et Pascal}
}
\newglossaryentry{Merge Request}
{
    name=Merge Request : ,
    description={(aussi appelé demande de fusion) Déclaration d'une demande de fusion entre le code nouvellement développé et le code déjà présent sur le répertoire (Git)}
}
\newglossaryentry{gRPC}
{
    name=gRPC : ,
    description={gRPC est un framework RPC (Remote procedure call) open source initialement développé par Google. Le RPC, Remote procedure call, est un protocole réseau permettant de faire des appels de procédures sur un ordinateur distant àl’aide d’un serveur d’applications}
}
\newglossaryentry{RollBack}
{
    name=Rollback : ,
    description={(en français le retour arrière) Désigne le fait de revenir sur la version de l'application précédemment déployée}
}
\newglossaryentry{CloudNative}
{
    name=CloudNative : ,
    description={ Le "Cloud native" est une approche permettant de construire et d'exécuter des applications qui exploitent pleinement les avantages du modèle de l'informatique dans le cloud. Les technologies telles que les conteneurs, les micro-services, les APIs REST répondent a ce besoin}
}
\newglossaryentry{Stateless}
{
    name=Stateless : ,
    description={Une application est dîte stateless (ou sans état) si aucune des données échangées avec un client n'est stockées en session (dans le cadre d'un futur échange)}
}
\newglossaryentry{Cluster}
{
    name=Cluster : ,
    description={Un cluster  (ferme de serveur) est un ensemble de nœuds (nodes) qui exécutent des applications conteneurisées gérées par un orchestrateur}
}
\newglossaryentry{Namespace}
{
    name=Namespace : ,
    description={Un namespace est une partition virtuelle d'un cluster Kubernetes associée à un groupe d'utilisateur}
}

\newglossaryentry{Pipeline}
{
    name=Pipeline. : ,
    description={Au sein de Gitlab, un pipeline est un ensemble de tâches s'exécutant les unes à la suite des autres}
}


\newglossaryentry{Hyperviseur}
{
    name=Hyperviseur : ,
    description={L’hyperviseur est un applicatif tournant sur l’hôte qui va émuler (ou simuler) le matériel pour une machine virtuelle}
}

\newglossaryentry{Binaires}
{
    name=Binaires : ,
    description={Les binaires désignent tous les fichiers qui ne sont pas interprétables sous forme de texte. Il peut s'agir de fichier compressé ou exécutable}
}

\newglossaryentry{Job}
{
    name=Job : ,
    description={Dans le cadre de Gitlab, un job est un ensemble de tâches qui ont lieu durant un pipeline. Dans le cadre de Kubernetes, un job désigne le fait de déployer une application dans l'orchestrateur.}
}

\newglossaryentry{JSON}
{
    name=JSON : ,
    description={JSON (JavaScript Object Notation) est un format de fichier dérivé de la notation des objets du langage JavaScript. Comme XML, ce format permet de représenter de l’information structurée}
}

\newglossaryentry{CLI}
{
    name=CLI : ,
    description={CLI signifie Command Line Interface (en français, interface en ligne de commande)}
}

\newglossaryentry{Requête POST}
{
    name=Requête POST : ,
    description={Une requête POST est une requête WEB permettant d'envoyer du contenu à un serveur}
}

\newglossaryentry{API}
{
    name=API : ,
    description={API signifie Application Programming Interface ou interface de programmation d’application. Une API est accessible par le Web et permet d’établir des connexions entre plusieurs applications pour échanger des données}
}

\newglossaryentry{hub}
{
    name=Hub : ,
    description={Plateforme regroupant l’ensemble des ressources sur un sujet précis au même endroit}
}
\newglossaryentry{patch}
{
    name=Patch : ,
    description={Correctif}
}
\newglossaryentry{Pair Programming}
{
    name=Pair Programming : ,
    description={le pair programming est une technique de développement logiciel agile dans laquelle deux programmeurs travaillent ensemble sur un même poste de travail. L’un, le driver, écrit le code tandis que l’autre, l’observateur, examine chaque ligne de code au fur et à mesure qu’elle est tapée. Les deux programmeurs changent fréquemment de rôle}
}

\newglossaryentry{Extreme Programming}
{
    name=Extreme Programming : ,
    description={Méthode de développement d'application consistant à développer une application par sprint très court.}
}
\newglossaryentry{YAML}
{
    name=YAML : ,
    description={YAML (Yet Another Markup Language) est un format de représentation de données.}
}
